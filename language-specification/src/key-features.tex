\section{Key Features}
\label{sec:key-features}

\subsection{Task}
\label{sec:task}
Tasks provide flexibility to represent various elements in a project, such as user stories or epics. Users can create tasks and structure them based on their needs and methodologies. This simplicity reduces the learning curve by minimizing the number of elements to master.

\subsection{List}
\label{sec:list}
Lists help structure and organize a project by dividing it into manageable parts. For example, in Scrum, a project can be separated into sprints, each represented as a list containing tasks. This flexibility allows adaptation to various project management methodologies.

\subsection{Body}
\label{sec:body}
The body of a task provides a comprehensive description and includes subtasks, breaking down complex tasks into smaller ones. This ensures all relevant information is documented.

\subsection{Status}
\label{sec:status}
Statuses indicate the progress of tasks within the workflow, such as ToDo, In Progress, and Done. They add visibility and facilitate the management of the task lifecycle, ensuring systematic progress through different stages.

\subsection{Function}
\label{sec:function}
Functions allow operations on tasks or lists, such as creation, editing, or verification. Creating or editing tasks involves specifying necessary parameters, preserving the characteristics of a functional language.

\subsection{Cycle}
\label{sec:cycle}
A loop to perform operations on tasks or lists while a condition is met. In our functional language, a while loop can automate repetitive tasks, such as adding sprints in a Scrum project.

\subsection{Conditionals}
\label{sec:conditionals}
Conditionals add clarity and expressivity to the DSL, enabling decision-making based on specific criteria. They help manage the flow of execution and define automated rules responding to project state changes.

