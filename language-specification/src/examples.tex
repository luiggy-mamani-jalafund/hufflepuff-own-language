\section{Example}
\label{sec:bnf}
\subsection {Project Description}
This is a project to create a Farm Automatization application. It's purpose is to replace repetitive tasks a farmer would do.
The team is composed by:
\begin{itemize}
	\item "Alice", "Soil Scientist"
	\item "Bob", "IoT Specialist"
	\item "Charlie", "Meteorologist"
	\item "Dana", "Backend Developer"
	\item "Eve", "Agronomist"
	\item "Frank", "Frontend Developer"
	\item "Grace", "Hydrologist"
	\item "Hank", "Embedded Systems Developer"
	\item "Ivy", "Plant Pathologist"
	\item "Jack", "AI Specialist"
	\item "Kate", "Inventory Manager"
	\item "Leo", "Database Developer"
	\item "Mia", "Operations Manager"
	\item "Nate", "Backend Developer"
	\item "Olivia", "Data Scientist"
	\item "Paul", "Full Stack Developer"
	\item "Quinn", "Security Specialist"
	\item "Riley", "Backend Developer"
\end{itemize}

The most important features are:
\begin{itemize}
	\item Design the schema for storing information about the vegetables, soil conditions, and growth stages.
	\item Develop the software interface to communicate with soil and weather sensors.
	\item Create the web application for farmers to monitor and control the cultivation process.
	\item Develop the functionality to measure and report soil moisture, pH, and nutrient levels in real-time.
	\item Implement weather tracking to monitor local conditions and predict weather patterns.
	\item Develop scheduling tools to optimize planting and harvesting cycles, including crop rotation.
	\item Automate irrigation based on soil and weather data, with manual control options.
	\item Identify common pests and diseases, and suggest treatments and preventive measures.
	\item Track inventory of seeds, fertilizers, pesticides, and supplies, and provide low stock alerts.
\end{itemize}

\subsection {Project represented as code}
\lstinputlisting {../../grammar/syntax_example.hp}
