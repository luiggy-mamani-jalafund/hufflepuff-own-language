\documentclass{article}
\usepackage{fancyhdr} 
\usepackage{graphicx} 
\usepackage{hyperref}

\title{Specification Document}
\author{Hufflepuff}
\date{June 2024}

\pagestyle{fancy}
\fancyhf{} 
\fancyhead[L]{\includegraphics[width=2cm]{images/jala-logo.png}}

\begin{document}

\maketitle

\tableofcontents
\newpage

\section{Introduction}
\label{sec:introduction}
This document describes the project specifications. A domain-specific language based on the Extended Backus-Naur Form (EBNF) grammar. It is designed to facilitate project management and development using the SCRUM framework. Defines constructs to represent projects, teams, tasks, lists, functions, loops, conditionals, and other building blocks necessary for SCRUM project management.

\section{Purpose}
\label{sec:purpose}
The main objective of this document is to provide a detailed description of the project specifying all aspects are well documented.
\\\\
Our language seeks to facilitate the management of a software project regardless of the necessary methodology, such as Scrum, Kanban, etc. In the world of software development, agile methodologies are accepted due to their flexible and customer-centric approach. While methodologies like Scrum and Kanban each have their unique strengths, effective project management often requires tools that transcend the limitations of any single methodology.
\\\\
To facilitate this complexity, we have developed a versatile Domain Specific Language (DSL) designed to provide a specialized tool that allows teams to describe, manage and execute their projects more efficiently, regardless of the specific methodology employed. 
\\\\
So, this DSL aims to be a robust and adaptable tool for project management, improving the efficiency and effectiveness of software development processes through clear, precise and methodology-independent specifications.

\section{Key Features}
\label{sec:key-features}
% Your content here

\section{Compiled or Interpreter}
\label{sec:compiled-or-interpreter}
% Your content here

\section{EBNF}
\label{sec:bnf}
% Content EBNF here

\section{Railroad diagram}
\label{sec:grail-road}
% Content Rail here

\section{Team members}
\label{sec:members}
\begin{itemize}
    \item \textbf{Axel Javier Ayala Siles}
    \item \textbf{Diego Hernan Figueroa Sevillano}
    \item \textbf{Gabriel Santiago Concha Saavedra}
    \item \textbf{Leonardo Herrera Rosales}
    \item \textbf{Luiggy Mamani Condori}
\end{itemize}

\end{document}
