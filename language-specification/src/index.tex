\documentclass{article}
\usepackage{fancyhdr} 
\usepackage{graphicx} 
\usepackage{hyperref}

\title{Specification Document}
\author{Hufflepuff}
\date{June 2024}

\pagestyle{fancy}
\fancyhf{} 
\fancyhead[L]{\includegraphics[width=2cm]{images/jala-logo.png}}

\begin{document}

\maketitle

\tableofcontents
\newpage

\section{Introduction}
\label{sec:introduction}
This document provides a complete specification for the project, detailing the use of a domain-specific language (DSL) based on the Extended Backus-Naur Form (EBNF) grammar. The DSL is designed to facilitate the management and development of software projects, regardless of the specific methodology used, such as Scrum, Kanban or other agile frameworks. Defines constructs to represent key elements necessary for effective project management, including projects, teams, tasks, lists, functions, loops, conditionals, and other essential components. By providing a structured and adaptable language, this DSL aims to improve clarity, reduce complexity and improve the overall efficiency of project management processes.

\section{Purpose}
\label{sec:purpose}
The main objective of this document is to provide a detailed description of the project specifying all aspects are well documented.
\\\\
Our language seeks to facilitate the management of a software project regardless of the necessary methodology, such as Scrum, Kanban, etc. In the world of software development, agile methodologies are accepted due to their flexible and customer-centric approach. While methodologies like Scrum and Kanban each have their unique strengths, effective project management often requires tools that transcend the limitations of any single methodology.
\\\\
To facilitate this complexity, we have developed a versatile Domain Specific Language (DSL) designed to provide a specialized tool that allows teams to describe, manage and execute their projects more efficiently, regardless of the specific methodology employed. 
\\\\
So, this DSL aims to be a robust and adaptable tool for project management, improving the efficiency and effectiveness of software development processes through clear, precise and methodology-independent specifications.

\section{Key Features}
\label{sec:key-features}
% content here

\section{Compiled or Interpreter}
\label{sec:compiled-or-interpreter}
% content here

\section{EBNF}
\label{sec:bnf}
% Content EBNF here

\section{Railroad diagram}
\label{sec:grail-road}
% Content Rail here

\section{Team members}
\label{sec:members}
\begin{itemize}
    \item \textbf{Axel Javier Ayala Siles}
    \item \textbf{Diego Hernan Figueroa Sevillano}
    \item \textbf{Gabriel Santiago Concha Saavedra}
    \item \textbf{Leonardo Herrera Rosales}
    \item \textbf{Luiggy Mamani Condori}
\end{itemize}

\end{document}
